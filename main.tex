\documentclass{IEEEtran}

\usepackage[margin=0.5in]{geometry}
\usepackage{graphicx}
\usepackage{titling}

\begin{document}
\title{
  Neuroscience: Cellular Circuit Mechanisms \\
  Mini-Project, BIO-482, EPFL
}
\author{
  Wesley Monteith-Finas, SCIPER: 324745
  % \\ December 2023
}
\maketitle

\section{Properties of cortical neurons during quiet wakefulness}

Question 1 (1/10 marks):
\begin{enumerate}
  \item Based on what you have learned during the course, explain what could be the impact of the AP threshold, the mean Vm and the SD of the Vm on the mean firing rate of a neuron. 
  \item 
  Based on the analyses performed in Part 1, identify which property(ies) actually influence the mean firing rate of cortical neurons across cell-classes? Justify your answer with some graphs.
\end{enumerate}

\subsection{Suprathreshold activity}

\subsection{Subthreshold activity}

\section{Membrane potential dynamics and motor activity}

Question 2 (2/10 marks):
\begin{enumerate}
  \item What are the specificities of each class of cortical neurons allowing to best distinguish excitatory vs inhibitory neurons?
  \item And between the different subclasses of inhibitory neurons?
  \item Justify your answers with some graphs.
\end{enumerate}


\section{Sensory evoked neuronal activity}

Question 3 (1/10 mark):

\begin{enumerate}
  \item Summarize what happens at whisking onset time and active-contact onset time for the different cell-classes. Justify your answers with some graphs.
\end{enumerate}

\section{Personal project}

\begin{enumerate}
  \item Explain in a few lines what is the question you want to address, what is the rational, and what is your hypothesis?
  \item Explain briefly what analyses you have done to answer your question and how you have proceeded.
  \item Present your results with some graphs and explanations.
  \item Interpret your results, answer your question if possible or explain why you cannot, conclude.
\end{enumerate}


\end{document}